\section{Generalized Regular Expressions}
This section outlines the method for building expressions which define
regular languages.
This corresponds to the one-tape automaton case for INR.
In these situations, INR will compute a minimized deterministic finite
automaton (DFA MIN) and display the result.
More mathematically inclined readers will notice an anomolous use of set
notation that is common in the formal language theory literature and
continued here to avoid pedantry and cumbersome notational problems.
A string containing one letter is ambiguously written as the letter and a
set containing one element is ambiguously written as its element.
With this understanding \verb#a# $\equiv$ \verb#{a}# $\equiv$ \verb#{'a'}#.

\subsection{Token}
A token is a primitive element.
It is represented by a symbol that has not been assigned a value as a
variable and denotes the automaton that accepts that token and then quits.
\begin{center}\begin{minipage}[t]{3in}\begin{minipage}[t]{3in}\begin{tabbing}
\qquad \= \verb#abc;#\\
or \> \verb#`abc`;#
\end{tabbing}\end{minipage}\end{minipage}
\begin{minipage}[t]{1.6in}\begin{verbatim}
(START) abc 2
2 -| (FINAL)
\end{verbatim}\end{minipage}\end{center}

\subsection{Variable}
A variable is a symbol that has been previously been assigned a value using
an assignment statement.
\begin{center}\begin{minipage}[t]{3in}\begin{minipage}[t]{3in}\begin{tabbing}
\qquad \= \verb#var = abc; var;#
\end{tabbing}\end{minipage}\end{minipage}
\begin{minipage}[t]{1.6in}\begin{verbatim}
(START) abc 2
2 -| (FINAL)
\end{verbatim}\end{minipage}\end{center}
Variables should use names that are clearly distinguishable from tokens.
For example, they should be at least two letters in length to avoid
conflict with token names.

\subsection{String}
A string is a possibly empty sequence of tokens and denotes the automaton
that recognizes only that sequence.
Strings are indicated by writing the tokens in the sequence in which they
occur.
They normally require spaces between adjacent tokens because of the
lexical rules for distinguishing symbols.
Thus \verb#a b c# indicates a string with three one-letter tokens whereas
\verb#abc# indicates a single token.

Since one-letter tokens are very useful to denote the printable characters
of the ASCII (or EBCDIC) character set an alternative notation is provided
to indicate strings of one-letter tokens.
Any such string can be written without blanks but preceded and followed
by single forward quotes ('):
\begin{center}\begin{minipage}[t]{3in}\begin{minipage}[t]{3in}\begin{tabbing}
\qquad \= \verb#a b c;#\\
or \> \verb#'abc';#
\end{tabbing}\end{minipage}\end{minipage}
\begin{minipage}[t]{1.6in}\begin{verbatim}
(START) a 2
2 b 3
3 c 4
4 -| (FINAL)
\end{verbatim}\end{minipage}\end{center}

Any sequence of characters including operators may be used within a string 
with the following exceptions: \verb#\'# indicates a single forward
quote character, \verb#\\# indicates a single backslash character,
\verb#\n# indicates a newline character, and \verb#\t# indicates a tab
character.
Note that strings of one character are a good way of entering single letter
tokens that are otherwise treated as operators or white space by the
lexical scanner.
Thus the token containing exactly one blank can be specified using the
notation \verb#' '#, newline can be specified by \verb#'\n'#, and the plus
character by \verb#'+'#.
Although backquotes also work in these cases, a greater uniformity of
notation can be achieved if many strings also occur.

\subsection{Empty String}
The empty string is indicated by \verb#^#.
This symbol is used because of its vague similariy to the Greek letter
$\Lambda$ that is often used for this purpose.
Note that in EBCDIC this character has a graphic representation of
$\not c$.
Alternative representations is \verb#()# (empty parentheses)
or \verb#''# (empty forward quotes).
\begin{center}\begin{minipage}[t]{3in}\begin{minipage}[t]{3in}\begin{tabbing}
\qquad \= \verb#^;#\\
or \> \verb#();#\\
or \> \verb#'';#
\end{tabbing}\end{minipage}\end{minipage}
\begin{minipage}[t]{1.6in}\begin{verbatim}
(START) -| (FINAL)
\end{verbatim}\end{minipage}\end{center}

\subsection{Finite Set of Strings}
Finite sets of strings are indicated using the standard brace/comma notation.
Thus \verb#{'ab','aa','a'}# denotes an automaton recognizing a three
strings \verb#'ab'#, \verb#'aa'# and \verb#'a'#.
\begin{center}\begin{minipage}[t]{3in}\begin{minipage}[t]{3in}\begin{tabbing}
\qquad \= \verb#{'ab','aa','a'};#
\end{tabbing}\end{minipage}\end{minipage}
\begin{minipage}[t]{1.6in}\begin{verbatim}
(START) a 2
2 -| (FINAL)
2 a 3
2 b 3
3 -| (FINAL)
\end{verbatim}\end{minipage}\end{center}

\subsection{Empty Set}
The empty set is indicated by \verb#{}#.
\begin{center}\begin{minipage}[t]{3in}\begin{minipage}[t]{3in}\begin{tabbing}
\qquad \= \verb#{};#
\end{tabbing}\end{minipage}\end{minipage}
\begin{minipage}[t]{1.6in}\begin{verbatim}
Empty Automaton
\end{verbatim}\end{minipage}\end{center}

\subsection{Alternation (Set Union)}
The operator for alternation (OR) is expressed using the vertical bar
\verb#|#.
\begin{center}\begin{minipage}[t]{3in}\begin{minipage}[t]{3in}\begin{tabbing}
\qquad \= \verb#{'abc'}|{'def'};#
\end{tabbing}\end{minipage}\end{minipage}
\begin{minipage}[t]{1.6in}\begin{verbatim}
(START) a 2
(START) d 3
2 b 4
3 e 5
4 c 6
5 f 6
6 -| (FINAL)
\end{verbatim}\end{minipage}\end{center}
Note that the concatenations are done before the alternation since
concatenation has a higher priority than alternation.
Parentheses may be used to enforce a different order of evaluation:
\begin{center}\begin{minipage}[t]{3in}\begin{minipage}[t]{3in}\begin{tabbing}
\qquad \= \verb#a b (c | d) e f;#\\
or \> \verb#a b c e f | a b d e f;#
\end{tabbing}\end{minipage}\end{minipage}
\begin{minipage}[t]{1.6in}\begin{verbatim}
(START) a 2
2 b 3
3 c 4
3 d 4
4 e 5
5 f 6
6 -| (FINAL)
\end{verbatim}\end{minipage}\end{center}

Note also that the brace-comma set notation is always equivalent to
parentheses with \verb#|#.
This departs slightly from standard set theory but is a natural
generalization in this situation.
Thus the above automata can also be denoted by
\verb#a b { c, d } e f#.

\subsection{Concatenation}
Concatenation is represented by adjacency.
That is, there is no explicit concatenation operator although \verb#^# can
be used if desired.
\begin{center}\begin{minipage}[t]{3in}\begin{minipage}[t]{3in}\begin{tabbing}
\qquad \= \verb#{ a, b } { c, d e };#\\
or \> \verb#{ a, b }^{ c, d e };#\\
or \> \verb#{'ac','ade','bc','bde'};#
\end{tabbing}\end{minipage}\end{minipage}
\begin{minipage}[t]{1.6in}\begin{verbatim}
(START) a 2
(START) b 2
2 c 3
2 d 4
3 -| (FINAL)
4 e 3
\end{verbatim}\end{minipage}\end{center}
Note that strings, as discussed before, are just the concatenation of their
constituent tokens.

\subsection{Kleene Plus and Star}
The concatenation closure operators (Kleene $+$ and Kleene $*$) are represented
by the corresponding unary postfix operator.
\begin{center}\begin{minipage}[t]{3in}\begin{minipage}[t]{3in}\begin{tabbing}
\qquad \= \verb#(a b)+;#\\
or \> \verb#'ab'+;#
\end{tabbing}\end{minipage}\end{minipage}
\begin{minipage}[t]{1.6in}\begin{verbatim}
(START) a 2
2 b 3
3 -| (FINAL)
3 a 2
\end{verbatim}\end{minipage}\end{center}
\begin{center}\begin{minipage}[t]{3in}\begin{minipage}[t]{3in}\begin{tabbing}
\qquad \= \verb#(a b)*;#\\
or \> \verb#'ab'*;#
\end{tabbing}\end{minipage}\end{minipage}
\begin{minipage}[t]{1.6in}\begin{verbatim}
(START) -| (FINAL)
(START) a 2
2 b (START)
\end{verbatim}\end{minipage}\end{center}
The difference is that the second form includes the empty word.
Note that parentheses are used to group together the argument to + and *.
This is because these two operators are considered to have higher priority
than any of the others.
\begin{center}\begin{minipage}[t]{3in}\begin{minipage}[t]{3in}\begin{tabbing}
\qquad \= \verb#a b*;#\\
or \> \verb#a (b*);#
\end{tabbing}\end{minipage}\end{minipage}
\begin{minipage}[t]{1.6in}\begin{verbatim}
(START) a 2
2 -| (FINAL)
2 b 2
\end{verbatim}\end{minipage}\end{center}

\subsection{Optional}
The operator indicating zero or one occurrences of a set is denoted by
\verb#?#.
This can be used to identify optional components.
\begin{center}\begin{minipage}[t]{3in}\begin{minipage}[t]{3in}\begin{tabbing}
\qquad \= \verb#a b?;#
\end{tabbing}\end{minipage}\end{minipage}
\begin{minipage}[t]{1.6in}\begin{verbatim}
(START) a 2
2 -| (FINAL)
2 b 3
3 -| (FINAL)
\end{verbatim}\end{minipage}\end{center}
The operators \verb#+#, \verb#*#, and \verb#?# are computed from left to
right.

\subsection{Concatenation Power}
To express the concatenation of a set with itself exactly $k$ times the
expression may be followed by a \verb#:# and the value $k$.
Thus \verb#R1 :2# is the same as \verb#R1 R1# and \verb#R1 :3# the same as
\verb#R1 R1 R1#.
\begin{center}\begin{minipage}[t]{3in}\begin{minipage}[t]{3in}\begin{tabbing}
\qquad \= \verb#{a,b} :2;#
\end{tabbing}\end{minipage}\end{minipage}
\begin{minipage}[t]{1.6in}\begin{verbatim}
(START) a 2
(START) b 2
2 a 3
2 b 3
3 -| (FINAL)
\end{verbatim}\end{minipage}\end{center}
To express the concatenation of a set with itself $k$ or fewer times the
notation \verb#R1? :#$k$ can be used.
To express the concatenation of a set with itself more than $k$ times the
set difference can be used: \verb#R1* - (R1? :#$k$\verb#)#.

\subsection{Concatenation Quotient}
If concatenation is interpreted as a multiplication of strings so that
\verb#'abc'# $\cdot$ \verb#'def'# $=$ \verb#'abcdef'# we can define a right
quotient operator $/$ as \verb#'abcdef'# / \verb#'def'# $=$ \verb#'abc'#
and a left quotient operator $\backslash$ as \verb#'abc'# $\backslash$
\verb#'abcdef'# $=$ \verb#'def'#.
The (left or right) quotient of sets is defined as the union of their
elementwise quotients.
\begin{center}\begin{minipage}[t]{3in}\begin{minipage}[t]{3in}\begin{tabbing}
\qquad \= \verb#'ab' \ 'abcdef' / 'ef';#\\
or \> \verb#'cd';#
\end{tabbing}\end{minipage}\end{minipage}
\begin{minipage}[t]{1.6in}\begin{verbatim}
(START) c 2
2 d 3
3 -| (FINAL)
\end{verbatim}\end{minipage}\end{center}

\subsection{Set Intersection}
The automaton recognizing the intersection of two regular languages is
indicated by the \verb#&# operator.
\begin{center}\begin{minipage}[t]{3in}\begin{minipage}[t]{3in}\begin{tabbing}
\qquad \= \verb#{a,b}* & b {a,c}*;#\\
or \> \verb#b a*;#
\end{tabbing}\end{minipage}\end{minipage}
\begin{minipage}[t]{1.6in}\begin{verbatim}
(START) b 2
2 -| (FINAL)
2 a 2
\end{verbatim}\end{minipage}\end{center}

\subsection{Set Difference}
The operator \verb#-# indicates the difference of two two regular
languages.
\begin{center}\begin{minipage}[t]{3in}\begin{minipage}[t]{3in}\begin{tabbing}
\qquad \= \verb#{a,b}* - b a*;#\\
or \> \verb#(^ | a | b a* b) {a,b}*;#
\end{tabbing}\end{minipage}\end{minipage}
\begin{minipage}[t]{1.6in}\begin{verbatim}
(START) -| (FINAL)
(START) a 2
(START) b 3
2 -| (FINAL)
2 a 2
2 b 2
3 a 3
3 b 2
\end{verbatim}\end{minipage}\end{center}

\subsection{Symmetric Difference (Exclusive Or)}
The symmetric difference of two languages is the set of words that are in
one of the languages but not the other and will be indicated by the
\verb#!# operator.
Thus $R_1 ! R_2 \equiv (R_1-R_2) \cup (R_2-R_1)$.
\begin{center}\begin{minipage}[t]{3in}\begin{minipage}[t]{3in}\begin{tabbing}
\qquad \= \verb#{a,b}* ! {a,c}*;#\\
or \> \verb#a* ( b {a,b}* | c {a,c}* );#
\end{tabbing}\end{minipage}\end{minipage}
\begin{minipage}[t]{1.6in}\begin{verbatim}
(START) a (START)
(START) b 2
(START) c 3
2 -| (FINAL)
2 a 2
2 b 2
3 -| (FINAL)
3 a 3
3 c 3
\end{verbatim}\end{minipage}\end{center}
Note that the symmetric difference of a language with itself is the empty
set.
Thus this operation can be used to check the equivalence of two regular
languages.

\subsection{Shuffle}
The shuffle of two words is the set of words formed by alternatively taking
some letters from each of the words.
Thus the shuffle of the words \verb#'ab'# and \verb#'cd'# is the set
\verb#{'abcd','acbd','acdb','cabd','cadb','cdab'}#.
The shuffle of two sets of words is the union of the shuffles of all pairs
of words such that one word is taken from each set.
The shuffle operator is indicated by the symbol \verb#!!# which mildly
suggests the symbol $\amalg$ that is often used for this purpose.
\begin{center}\begin{minipage}[t]{3in}\begin{minipage}[t]{3in}\begin{tabbing}
\qquad \= \verb#'ab' !! 'cd';#\\
or \> \verb#{'abcd','acbd','acdb',#\\
\> \verb#  'cabd','cadb','cdab'};#
\end{tabbing}\end{minipage}\end{minipage}
\begin{minipage}[t]{1.6in}\begin{verbatim}
(START) a 2
(START) c 3
2 b 4
2 c 5
3 a 5
3 d 6
4 c 7
5 b 7
5 d 8
6 a 8
7 d 9
8 b 9
9 -| (FINAL)
\end{verbatim}\end{minipage}\end{center}

\subsection{Active Alphabet}
The active alphabet of an automation is the set tokens that can actually
occur in a word in the language.
\begin{center}\begin{minipage}[t]{3in}\begin{minipage}[t]{3in}\begin{tabbing}
\qquad \= \verb#'abcd' / 'd' :alph;#\\
or \> \verb#{a,b,c};#
\end{tabbing}\end{minipage}\end{minipage}
\begin{minipage}[t]{1.6in}\begin{verbatim}
(START) a 2
(START) b 2
(START) c 2
2 -| (FINAL)
\end{verbatim}\end{minipage}\end{center}

\subsection{Reverse}
The reverse of a string \verb#'abcd'# is the string \verb#'dcba'#.
The reverse of a set of strings is the set of their reverses.
\begin{center}\begin{minipage}[t]{3in}\begin{minipage}[t]{3in}\begin{tabbing}
\qquad \= \verb#{'abc','abd'} :rev;#\\
or \> \verb#{'dba','cba'};#
\end{tabbing}\end{minipage}\end{minipage}
\begin{minipage}[t]{1.6in}\begin{verbatim}
(START) c 2
(START) d 2
2 b 3
3 a 4
4 -| (FINAL)
\end{verbatim}\end{minipage}\end{center}

\subsection{Prefixes}
The prefixes of \verb#'abcd'# is the set
\verb#{ ^, 'a', 'ab', 'abc', 'abcd' }#.
The prefixes of a set of strings is the union of the prefix sets for each
word.
Note that \verb#R1 :pref# is an alternate (more efficient) way of indicating
the set \verb#R1 / (R1 :alph)*#.
\begin{center}\begin{minipage}[t]{3in}\begin{minipage}[t]{3in}\begin{tabbing}
\qquad \= \verb#{'abc','abd'} :pref;#\\
or \> \verb#{'abc','abd'} / {a,b,c,d}*;#\\
or \> \verb#{^,'a','ab','abc','abd'};#
\end{tabbing}\end{minipage}\end{minipage}
\begin{minipage}[t]{1.6in}\begin{verbatim}
(START) -| (FINAL)
(START) a 2
2 -| (FINAL)
2 b 3
3 -| (FINAL)
3 c 4
3 d 4
4 -| (FINAL)
\end{verbatim}\end{minipage}\end{center}

\subsection{Suffixes}
The suffixes of \verb#'abcd'# is the set
\verb#{ ^, 'd', 'cd', 'bcd', 'abcd' }#.
The suffixes of a set of strings is the union of the suffix sets for each
word.
Note that \verb#R1 :suff# is an alternate way of indicating
the set \verb#(R1 :alph)* \ R1#.
\begin{center}\begin{minipage}[t]{3in}\begin{minipage}[t]{3in}\begin{tabbing}
\qquad \= \verb#{'abc','abd'} :suff;#\\
or \> \verb#{a,b,c,d}* \ {'abc','abd'};#\\
or \> \verb#{^,'c','d','bc','bd',#\\
\> \verb#  'abc','abd'};#
\end{tabbing}\end{minipage}\end{minipage}
\begin{minipage}[t]{1.6in}\begin{verbatim}
(START) -| (FINAL)
(START) a 2
(START) b 3
(START) c 4
(START) d 4
2 b 3
3 c 4
3 d 4
4 -| (FINAL)
\end{verbatim}\end{minipage}\end{center}
The factors (subwords) of a language can be obtained by using finding the
prefixes of the suffixes: \verb#R1 :suff :pref#.
For large sets this is preferable to the equivalent but much more expensive
\verb#R1 :pref :suff#.

\subsection{Complement}
Since the alphabet of interest is inferred from the context, the complement
must be defined with respect to a specific universe.
Thus there are two basic forms \verb#R1 :acomp# and \verb#R1 :comp#.
The first of these is equivalent to \verb#(R1 :alph)* - R1# and the second
is equivalent to \verb#(SIGMA :alph)* - R1# where \verb#SIGMA# contains the
reference alphabet.
\begin{center}\begin{minipage}[t]{3in}\begin{minipage}[t]{3in}\begin{tabbing}
\qquad \= \verb#b a* :acomp;#\\
or \> \verb#SIGMA = {a,b}; b a* :comp;#\\
or \> \verb#{a,b}* - b a*;#\\
or \> \verb#(^ | a | b a* b) {a,b}*;#
\end{tabbing}\end{minipage}\end{minipage}
\begin{minipage}[t]{1.6in}\begin{verbatim}
(START) -| (FINAL)
(START) a 2
(START) b 3
2 -| (FINAL)
2 a 2
2 b 2
3 a 3
3 b 2
\end{verbatim}\end{minipage}\end{center}

\subsection{Stretching}
An operation that is sometimes useful involves replacing each letter in the
word by a sequence of $k$ of that letter.
This can be done using the \verb#$# operator followed by a string of $k$
zeros.
\begin{center}\begin{minipage}[t]{3in}\begin{minipage}[t]{3in}\begin{tabbing}
\qquad \= \verb#'abc' $ 0 0;#\\
or \> \verb#'aabbcc';#
\end{tabbing}\end{minipage}\end{minipage}
\begin{minipage}[t]{1.6in}\begin{verbatim}
(START) a 2
2 a 3
3 b 4
4 b 5
5 c 6
6 c 7
7 -| (FINAL)
\end{verbatim}\end{minipage}\end{center}

\subsection{Priorities of Operators}
Within a pair of matching parentheses or braces the expression is computed in
the following order:
\begin{enumerate}
\item Kleene $+$, Kleene $*$ and Optional operators are evaluated from left
to right on their single argument.
\item Concatenations
\item One pair of matching left and right quotients are performed if they
exist.
\item Set intersection and set difference operations are performed from
left to right.
\item Alternation (set union), symmetric difference, and shuffle operations
are performed from left to right.
\item One stretch operation
\item The colon unary operations concatenation power, :alph, :rev, :pref, and
:suff are performed left to right.
\item The alternation implied by commas within braces is performed last.
\end{enumerate}
Note that the colon unary operations will often require surrounding
parentheses in a larger expression.

Note that parentheses or braces can be used to perform grouping when the
priorities of operators would imply an incorrect interpretation.
Use of parentheses is usually recommended unless a set constructor is
needed for clarity.
However, note that commas within parentheses have another meaning to be
discussed in a later section.
Another form of grouping is performed by string quotes.
Thus \verb#'ab'*# is equivalent to \verb#(a b)*# and not to
\verb#a b*#.
