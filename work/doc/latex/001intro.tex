\maketitle
\begin{abstract}
INR is a program that computes finite state automata and transducers from
provided regular and rational expressions.
\end{abstract}

\section{Introduction}
INR is a program that constructs finite state automata and transducers from
generalized rational expressions.
Although originally developed with to support research into
approximate string matching \cite{Johnson83} it has proved useful for a
number of other purposes.

The program operates in a traditional read/evaluate/print mode, that is, it
repeatedly reads an expression from the input stream, computes the
automaton required, and displays a result.
For example, if we wish to find the minimized automaton over the alphabet
$\{a,b\}$ that recognizes words that contain at least one ``$a$'' and at least
one ``$b$''.
This can be described using the extended regular expression indicating the
intersection of words containing $a$ and words containing $b$:
$$\{a,b\}^*a\{a,b\}^* \wedge \{a,b\}^*b\{a,b\}^*.$$
This expression can be presented to INR in the following form:
\begin{quote}
\begin{verbatim}
{a,b}* a {a,b}* & {a,b}* b {a,b}*;
\end{verbatim}
\end{quote}
INR will print back
\begin{quote}
\begin{verbatim}
DFA MIN States: 5      Trans: 9      Tapes: 1  Strg: 1 K

(START) a 2
(START) b 3
2 a 2
2 b 4
3 a 4
3 b 3
4 -| (FINAL)
4 a 4
4 b 4
\end{verbatim}
\end{quote}
indicating that a minimized deterministic one-tape finite automaton has been
computed with 5 states and 9 transitions.
It requires 1 Kbyte (1024 bytes) of memory in the representation used by INR.
After this the transitions of the automaton are presented in the form
\begin{quote}
{\it source-state \qquad label \qquad target-state}
\end{quote}
so that 
\begin{quote}
\begin{verbatim}
3 a 4
\end{verbatim}
\end{quote}
indicates a transition from state {\tt 3} to state {\tt 2} which reads an $a$.
The start state is denoted by the special state name {\tt (START)} and
final states are indicated by a transition on end-marker ({\tt -|}) to the
dummy state {\tt (FINAL)}.
In the above example
\begin{quote}
\begin{verbatim}
4 -| (FINAL)
\end{verbatim}
\end{quote}
indicates that {\tt 4} is a final state.

If we want to enumerate some of the words recognized by this automaton we
can specify the following command to INR:
\begin{quote}
\begin{verbatim}
{a,b}* a{a,b}* & {a,b}* b {a,b}*:;
\end{verbatim}
\end{quote}
The beginning of the result typed would then be:
\begin{quote}
\begin{verbatim}
a b 
b a 
a a b 
a b a 
a b b 
b a a 
b a b 
b b a 
a a a b 
a a b a 
a a b b 
a b a a 
a b a b 
a b b a 
a b b b 
b a a a 
 ...
\end{verbatim}
\end{quote}
The enumerator prints the first few words of the infinite sequence
terminating when it reaches a preset limit.

Another interactive use for INR involves testing whether two regular sets
are the same.
For example consider the set of all words over the alphabet $\{a,b\}^*$
that are not either all $a$'s or all $b$'s.
This can be denoted using the extended regular expression
$$\{a,b\}^* - ( \{a\}^* \cup \{b\}^* ).$$
This can be indicated to INR as:
\begin{quote}
\begin{verbatim}
{a,b}* - ( a* | b* );
\end{verbatim}
\end{quote}
To check that these are the same sets, we can compute the symmetric
difference of the two regular languages and verify that is empty.
The symmetric difference (exclusive or) operator for INR is the symbol
{\tt !}.
\begin{quote}
\begin{verbatim}
( {a,b}* a {a,b}* & {a,b}* b {a,b}* )
  ! ( {a,b}* - ( a* | b* ) );
\end{verbatim}
\end{quote}
This yields the result
\begin{quote}
\begin{verbatim}
Empty Automaton
\end{verbatim}
\end{quote}
indicating that the symmetric difference is empty, that is, that the two
indicated languages are identical.

In the first example a generalized regular expression was entered and an
automaton was printed back.
Instead of this, the automaton could have been written to a file ``temp''
in the form shown above:
\begin{quote}
\begin{verbatim}
{a,b}* a {a,b}* & {a,b}* b {a,b}* :pr temp;
\end{verbatim}
\end{quote}
Alternatively, it could be written to the file ``temp'' in a condensed
format which usually requires about half of the space and can be loaded more
quickly.
\begin{quote}
\begin{verbatim}
{a,b}* a {a,b}* & {a,b}* b {a,b}* :save temp;
\end{verbatim}
\end{quote}
It is also possible to assign it to an internal variable named ``temp'':
\begin{quote}
\begin{verbatim}
temp = {a,b}* a {a,b}* & {a,b}* b {a,b}*;
\end{verbatim}
\end{quote}
In each of these cases it is then available for computation in further
expressions allowing for a considerable saving in typing.
The value can be retrieved from the file temp in either {\tt :pr} form or
{\tt :save} form by means of the sub-expression {\tt :load temp}.
The value from a variable can be retrieved from an internal variable by
simply using the variable name itself (if that name has not been used as a
transition label).

INR is very often run in a non-interactive (batch) mode using variables to
build up larger and larger automata followed by a \verb#:pr# or a
\verb#:save# to store the result into a file.
For large automata this is the preferred approach since INR provides no
mechanisn for recalling past expressions or for editing them.
