\section{Statement Forms}
\subsection{Evaluate Statement}
An expression of the form
\verb#<expression>;#
requests the evaluation of the expression and the displaying of the
resulting automaton.
The variable \verb#_Last_# is updated by this statement.
If the last operation performed is a coercing operation, then the automaton
is printed.
If the last operation performed is not a coercing operation or a printing
operation, then the automaton is automatically coerced to \verb#:min# and
printed.
Thus the default is that a minimized deterministic automaton is printed.

\subsection{Enumerate Statement}
An expression of the form
\verb#<expression>:;#
requests the evaluation of the expression and the enumeration of the words
recognized by the resulting automaton.
The variable \verb#_Last_# is also updated by this statement.

\subsection{Assignment Statement}
Variables are simply symbols for which a value has been assigned.
Thus form of the assignment is:
\verb#<symbol> = <expression>;#
Any later references to the symbol will refer to the set assigned to it.
This is particularly useful for large expressions since often repeated
computations can be eliminated.

There is a built-in variable \verb#_Last_# which is assigned to the
last evaluated expression that was not directly assigned.
At any time a report on the assigned variables may be requested by the
command \verb#:list#.

\subsection{Command Statement}
A number of commands that cause certain actions to be performed are
provided.
\begin{quote}
\begin{describe}{XXXXXXXX}
\item[:alph;] Display token symbol table.
\item[:free;] Display status of free lists.
\item[:list;] Display variable symbol table.
\item[:noreport;] Turn off (verbose) debug tracing.
\item[:pr;] Save all variables in files with the same names.
\item[:quit;] Terminate session.
\item[:report;] Turn on (verbose) debug tracing.
\item[:save;] Save all variables in files (condensed) with the same names.
\item[:time;] Display time since last :time call (VAX only).
\end{describe}
\end{quote}
