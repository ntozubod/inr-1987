\section{Internals}
The internal workings of {\em INR} is achieved through the use of an
abstract data type whose values are automata.
A large number of operations are defined on objects of this type which
yield other automata.
Thus, for example, the operation union is implemented internally as a
function A\_union which accepts two automata as arguments and returns another.

In order to provide better access to the operations, an interface was
designed and implemented using ``yacc'', a LALR(1) parser generator
provided with UNIX.
The result is a user-friendly method of specifying transformations on
automata.
However, since the user commands are in the form of generalized rational
(or regular) expressions, the impression is that of translation from
expression form to automaton form.
However, there is no represention of expressions internally and even
strings and sets are coded as automata.

The form of the data structure is a ternary relation indicating the from
state, the transition label, and the to state.
Special states \verb#(START)# and \verb#(FINAL)# and special transitions
\verb#^^# and \verb#-|# simplify the processing.
The abstract data type also contains a mode indicator that indicates the
coercing level.
This allows the lazy evaluation mechanism to work and avoids a lot of
redundant computation.
